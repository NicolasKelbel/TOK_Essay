\documentclass[12pt]{article}

\usepackage[utf8]{inputenc}      
\usepackage[T1]{fontenc}          
\usepackage[a4paper, margin=2.5cm]{geometry}  % page size & margins
\usepackage{setspace}             % spacing 
\usepackage{graphicx}             % images
\usepackage{hyperref}             
\usepackage[backend=biber, style=apa]{biblatex} % citing
\usepackage{fontspec} % font
\usepackage[toc]{appendix}
\usepackage{float} % for figure positioning
\usepackage[font=small,labelfont=bf]{caption}
\usepackage[most]{tcolorbox}
\usepackage[noabbrev,nameinlink]{cleveref}
\usepackage{enumitem}
\usepackage{siunitx}


% Settings
\doublespacing
\addbibresource{references.bib}  

\setmainfont{Times New Roman}



\begin{document}
	
	
	\doublespacing
	
	\begin{titlepage}
		\centering
		
		\textbf{TOK Essay}
		
		\vspace*{4cm}
		
		\textbf{Prescribed Topic:}\\
		3. Is the power of knowledge determined by the way in which the knowledge is conveyed? Discuss with reference to mathematics and one other area of knowledge.
		
		\vspace{3cm}
		
		Word count: 1588\\
		
		
		\vspace{4cm}
		Candidate code: \textbf{lxv086}\\
		
		\vspace{2cm}
		Session: \textbf{May 2026}
		
		\vfill
	\end{titlepage}
		

Knowledge does not exercise power simply by being true. While knowledge may be acquired through individual inquiry, its ability to influence thought and action often depends on the means of exchange. Power is thus not intrinsic to knowledge itself, contingent instead upon whether it is usable and coherent to a particular audience. For the purpose of this essay, power will be examined in two distinct senses: epistemic power, the capacity of knowledge to shape what is accepted as justified belief, and social power, the intentional and effective influence exercised by knowledge over other members of society \parencites{Archer_Cawston_Matheson_Geuskens_2020, 141269df-8a9c-3d34-add1-77d740470f50}. The communication of mathematical knowledge predominately relies on formal proofs and symbolic representations, prioritizing logical validity over accessibility, whereas knowledge in the human sciences is conveyed through statistical representations and narratives, often constructed to persuade. This juxtaposition provides excellent grounds for discussion: while mathematical knowledge largely retains epistemic power regardless of conveyance, the social and epistemic power of knowledge in the human sciences are highly dependent on the method of transmission.

In mathematics, knowledge derives its epistemic power from proof, establishing validity regardless of the transmission method. Such proofs broadly lend themselves to two categories: explanatory proofs and proofs that establish validity \parencite{HannaGila1989}. Both warrant a conclusion, with the latter, counterintuitively, dominating mathematical practice, reflecting the primacy of the warranted conclusion over explanatory accessibility \parencite{Mizrahi2020-MIZPEA}. As a result, the epistemic power of mathematical knowledge remains stable across variations in conveyance, insofar as a single theorem may admit countless proofs employing different ways of knowing \textemdash\ namely reasoning, visual perception and intuition \textemdash\ yet still yield equivalent conclusions. This is aptly illustrated by the Pythagorean Theorem, for which over one hundred distinct proofs have been documented \parencite{Bogomolny_Pythagorean}. Particular compelling examples include the purely visual proofs provided in \textit{Proofs Without Words} by \textcite{nelsen1993proofs}, where the validity of mathematical conclusions is preserved despite the absence of symbolic or linguistic exposition. If the epistemic power of mathematical knowledge were contingent on conveyance, alterations in representations would affect whether the conclusion is valid \textemdash\ which mathematical reasoning does not exhibit. Nevertheless, epistemic power in mathematics consists not merely in logical correctness but in articulated proof within a formal system. The specific mode of communication does not inherently determine validity; yet without any form of demonstrable expression through which a claim can be examined and evaluated, validity cannot attain epistemic standing as recognized knowledge. Thus far, however, epistemic power has denoted only the potential capacity of knowledge rather than a practical force.

When considered in a social context, the power of mathematical knowledge is no longer exclusively secured by its validity; instead, it depends on its accessibility and relevance within a community, emerging only when knowledge transitions from personal understanding to shared comprehension. Where conveyance is ineffective, knowledge may prove socially inert despite holding epistemic rigor. Fermat's Last Theorem clearly is emblematic of this, remaining inaccessible to all save a small number of scholars \parencite{Cambridge_FLT_History}. While the conclusion is accepted as justified belief, its social influence is near negligible, given the proof's complex and intricate nature \parencite{Boston_FLT_2003}. As \textcite{STYLIANOU2002303} argues in his paper regarding \textit{Representational Negotiation}, the means by which knowledge in mathematics is conveyed is highly dependent on the extent to which it is understood, especially beyond expert audiences. It therefore follows that epistemic certainty of knowledge proves insufficient for social power, which is shaped by the way it is communicated. More radically, the formal complexity that underwrites the epistemic rigor of mathematical knowledge may simultaneously constrain its social reach, revealing a divergence between epistemic and social power. The influence of transmission on social power, however, remains conditional upon relevance \textemdash\ where a body of knowledge bears no practical, contextual or intellectual significance for a crowd, even effective transmission fails to generate social power. This insight does not negate the effect of conveyance; rather, it merely provides an extreme case in which the argument subtly wavers.

Mathematics reveals that epistemic power is mostly secured irrespective of how knowledge is conveyed \textemdash\ though it presupposes articulated proof \textemdash\ whereas social power is largely contingent on it. While neither relation is absolute, the identified exceptions serve to delimit rather than undermine their validity.

Rather than merely establish abstract truth, as is the case in mathematics, the human sciences are intrinsically embedded within the societies they seek to investigate and influence, and thus, cannot be meaningfully considered void of their environment. Whereas in mathematics the role of the onlooker is largely confined to verifying the antecedent conclusion, in the human sciences conclusions must be constructed by the individual through the interpretation and evaluation of empirical findings, rarely yielding single, absolute conclusions. As the uptake of information \textemdash\ and by extension the conclusion made and the knowledge assimilated \textemdash\ is shaped by how it is conveyed, both the epistemic and social power of knowledge in the human sciences are heavily reliant on the method of communication \parencite{Tversky1985}. This striking dependency became particularly visible during the COVID-19 pandemic: identical epidemiological data was communicated in manifold ways, underwriting differing claims and, in turn, generating a profound spectrum of public and political responses \parencite{Teschendorf09102024}. Consequently, both the epistemic and social power of the knowledge was molded by communicative representation, as public confidence in the credibility of information and the behavior of other members of the society was governed by the conveyance method. Moreover, to facilitate the communication of risk regarding the virus, the communicator must make axiological choices regarding the emphasis and framing of evidence \parencites{WuJohn2021RiskCommunication, Gigerenzer2007StatisticalLiteracy}. If these values fail to resonate with those of the recipient, the knowledge may manifest as latent, minimizing its social power. Hence, it is reasonable to conclude that the epistemic and social power of knowledge in the human sciences lies not simply in what is know, yet in how the knowledge is understood by the populace, highlighting its dependency on the communication method.  However, as mentioned previously, the capacity of the conveyance method to generate social power of knowledge ultimately depends on its relevance to the receiving individuals.  

Naturally, this inference raises existential concern, in particular due to the omnipresent nature of the human sciences in our lives. To mitigate this shortcoming, practitioners employ stringent methodological safeguards, seeking to stabilize knowledge claims beyond their initial method of transmission, calibrating its epistemic and social power. As demonstrated by the scandal concerning the austerity study \textit{Growth in a Time of Debt} by Reinhart and Rogoff's, the epistemic power of knowledge in the human sciences cannot simply be reduced to its initial method of communication. Arguing that economic growth declines sharply once public debt exceeds \SI{90}{\percent} of the Gross Domestic Product of a given nation, this study exerted a pervasive influence on macroeconomics policy debates worldwide, asserting substantial social power \parencite{10.1093/cje/bet075}. Nonetheless, subsequent independent replication exposed methodological bias in the original study, collapsing its established conclusion and eroding the epistemic force of the claim \textemdash\ and with it the social authority the study had exerted \textemdash\ demonstrating that in the human sciences epistemic power is not fixed by its original presentation but is perpetually recalibrated through institutionalized scholarly discourse, including peer review and published replications. Epistemic justification in this discipline rests upon methodological rigor and replicability, which constitute what counts as proof and thus render epistemic power dependent on ongoing processes of critique and validation. This suggests that in the human sciences, conveyance actively participates in constituting its epistemic power, as legitimacy emerges through iterative processes of public scrutiny. Yet, replication is not a purely neutral epistemic mechanism, rather a socially mediated practice contingent upon institutional priorities and funding structures. Subsequently, epistemic recalibration in the human sciences is uneven: persuasive conveyance may secure epistemic power for a claim, enabling it to function as justified knowledge, while that power may persist not through successful critique, but by never being subject to it. 

Subsequently, while communication initially confers social power upon knowledge in the human sciences, regulatory mechanisms can renegotiate its epistemic power and, thereby, its social power. Nevertheless, since such safeguards typically function after knowledge claims have been made, as holds for the above given example, this outcome prompts the question: can such retrospective corrections be considered to shape epistemic and social power despite only operating in hindsight?

To conclude, the exploration of mathematics and the human sciences has revealed that the power of knowledge is not uniformly determined by how it is conveyed. With reference to mathematics, epistemic power is grounded within articulated proof. Once secured, the mode of conveyance is predominantly inconsequential to its magnitude. By contrast, communication may starkly affect who is able to access, understand and or utilize the knowledge, thus affecting its social power. In the humans sciences, however, the epistemic and social power of knowledge is more closely bounded to the initial manner of transmission, as conclusions emerge through interpretive uptake rather than logical entailment. Yet, regulatory mechanism can later recalibrate this influence, albeit with uneven application. Across both areas of knowledge, the pertinence of knowledge to a given audience conditions its social power, independent of the method of imparting, while in the human sciences this relevance may further modulate its epistemic power. In light of the impossibility to quantify the the absolute dependence of the mode of conveyance, this essay argues that the power of knowledge may be dependent on the way it is conveyed, yet only under certain conditions. A possible extension of this discussion could involve investigating alternative conceptions of power, examining how these might alter the conclusion drawn about the influence of conveyance on the power of knowledge.





	
	
	
	

	
	
	
	
	
	
	\clearpage
	
	\printbibliography
	
	\clearpage
	
\end{document}