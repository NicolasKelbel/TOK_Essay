\documentclass[12pt]{article}

\usepackage[utf8]{inputenc}      
\usepackage[T1]{fontenc}          
\usepackage[a4paper, margin=2.5cm]{geometry}  % page size & margins
\usepackage{setspace}             % spacing 
\usepackage{graphicx}             % images
\usepackage{hyperref}             
\usepackage[backend=biber, style=apa]{biblatex} % citing
\usepackage{fontspec} % font
\usepackage[toc]{appendix}
\usepackage{float} % for figure positioning
\usepackage[font=small,labelfont=bf]{caption}
\usepackage[most]{tcolorbox}
\usepackage[noabbrev,nameinlink]{cleveref}
\usepackage{enumitem}


% Settings
\doublespacing
\addbibresource{references.bib}  

\setmainfont{Times New Roman}



\begin{document}
	
	
	\doublespacing
	
	\begin{titlepage}
		\centering
		
		\textbf{TOK Essay}
		
		\vspace*{4cm}
		
		\textbf{Prescribed Topic:}\\
		3. Is the power of knowledge determined by the way in which the knowledge is conveyed? Discuss with reference to mathematics and one other area of knowledge.
		
		\vspace{3cm}
		
		Word count:\\
		
		
		\vspace{4cm}
		Candidate code: \textbf{lxv086}
		
		\vfill
	\end{titlepage}
		
Knowledge does not exercise power simply by being true. While knowledge may be acquired through individual inquiry, its ability to influence thought and action often depends on the means of exchange. Power is thus not intrinsic to knowledge itself, contingent instead upon whether it is usable and coherent to a particular audience. For the purpose of this essay, power will be examined in two distinct senses: epistemic power, the capacity of knowledge to shape what is accepted as justified belief and credible, and social power, the intentional and effective influence exercised by knowledge over other members of society \parencites{Archer_Cawston_Matheson_Geuskens_2020, 141269df-8a9c-3d34-add1-77d740470f50}. The communication of Mathematical knowledge predominately relies on formal proofs and symbolic representations, prioritizing logical validity over accessibility, whereas knowledge in Human Sciences is conveyed through models, statistical representations and narratives, constructed to persuade. This juxtaposition provides excellent grounds for discussion: while Mathematical knowledge largely retains epistemic power regardless of conveyance, the social and epistemic power of knowledge in Human Sciences are highly dependent on the method of transmission.
	

Knowledge in Mathematics derives its epistemic power from proof, establishing validity regardless of the transmission method. Such proofs lend themselves to two categories: proofs that establish validity and proofs that provide explanation \parencite{HannaGila1989}. While perhaps counterintuitive, proofs that establish validity dominate mathematical practice, reflecting the primacy of the warranted conclusion over explanatory accessibility \parencite{Mizrahi2020-MIZPEA}. It is this preference which preserves epistemic power in mathematics \textemdash\ justification does not depend on audience comprehension, and consequently neither on the manner of communication. This separation from conveyance is evident when a single mathematical theorem admits countless proofs employing multiple ways of knowing \textemdash\ namely reasoning, visual perception and intuition \textemdash\ to establish epistemically equivalent mathematical results. This is aptly illustrated by the Pythagorean Theorem, a ubiquitous and transgenerational reference in mathematics. Over time, the number of distinct proofs of the theorem has augmented to exceed one hundred \parencite{Bogomolny_Pythagorean}. Drawing on a multitude of branches of mathematics, it can be established through both algebraic manipulation and visual rearrangement grounded in geometry. Particular compelling examples include the various proofs provided in the book \textit{Proofs Without Words} by \textcite{nelsen1993proofs}, demonstrating that the logical validity of mathematical conclusions is preserved despite the absence of conventional symbolic and linguistic modes of conveyance. Thus it is reasonable to conclude that the epistemic power of mathematical knowledge is autonomous from the communication method. Nevertheless, this raises the question whether this insulation of epistemic power implies that the manner in which the knowledge is transmitted is similarly inconsequential to its social power.

The social power of mathematical knowledge is not exclusively secured by its formal validity; instead, it depends on its perceived relevant within a community.


	
	
	
	

	
	
	
	
	
	
	\clearpage
	
	\printbibliography
	
	\clearpage
	
\end{document}