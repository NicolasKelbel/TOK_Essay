\documentclass[12pt]{article}

\usepackage[utf8]{inputenc}      
\usepackage[T1]{fontenc}          
\usepackage[a4paper, margin=2.5cm]{geometry}  % page size & margins
\usepackage{setspace}             % spacing 
\usepackage{graphicx}             % images
\usepackage{hyperref}             
\usepackage[backend=biber, style=apa]{biblatex} % citing
\usepackage{fontspec} % font
\usepackage[toc]{appendix}
\usepackage{float} % for figure positioning
\usepackage[font=small,labelfont=bf]{caption}
\usepackage[most]{tcolorbox}
\usepackage[noabbrev,nameinlink]{cleveref}
\usepackage{enumitem}


% Settings
\doublespacing
\addbibresource{references.bib}  

\setmainfont{Times New Roman}



\begin{document}
	
	
	\doublespacing
	
	\begin{titlepage}
		\centering
		
		\textbf{TOK Essay}
		
		\vspace*{4cm}
		
		\textbf{Prescribed Topic:}\\
		3. Is the power of knowledge determined by the way in which the knowledge is conveyed? Discuss with reference to mathematics and one other area of knowledge.
		
		\vspace{3cm}
		
		Word count:\\
		
		
		\vspace{4cm}
		Candidate code: \textbf{lxv086}
		
		\vfill
	\end{titlepage}
		

Knowledge does not exercise power simply by being true. While knowledge may be acquired through individual inquiry, its ability to influence thought and action often depends on the means of exchange. Power is thus not intrinsic to knowledge itself, contingent instead upon whether it is usable and coherent to a particular audience. For the purpose of this essay, power will be examined in two distinct senses: epistemic power, the capacity of knowledge to shape what is accepted as justified belief, and social power, the intentional and effective influence exercised by knowledge over other members of society \parencites{Archer_Cawston_Matheson_Geuskens_2020, 141269df-8a9c-3d34-add1-77d740470f50}. The communication of Mathematical knowledge predominately relies on formal proofs and symbolic representations, prioritizing logical validity over accessibility, whereas knowledge in Human Sciences is conveyed through models, statistical representations and narratives, constructed to persuade. This juxtaposition provides excellent grounds for discussion: while Mathematical knowledge largely retains epistemic power regardless of conveyance, the social and epistemic power of knowledge in Human Sciences are highly dependent on the method of transmission.

In Mathematics, knowledge derives its epistemic power from proof, establishing validity regardless of the transmission method. Such proofs broadly lend themselves to two categories: proofs that provide explanation and proofs that establish validity \parencite{HannaGila1989}. Crucially, both forms are sufficient as they warrant a conclusion, with the latter, counterintuitively, dominating mathematical practice, reflecting the primacy of the warranted conclusion over explanatory accessibility \parencite{Mizrahi2020-MIZPEA}. As a result, the epistemic power of mathematical knowledge remains stable across variations in conveying, evident in the fact that a single theorem may admit countless proofs employing different ways of knowing \textemdash\ namely reasoning, visual perception and intuition \textemdash\ yet still yield equivalent conclusions. This is aptly illustrated by the Pythagorean Theorem, a ubiquitous and transgenerational reference in mathematics. Over time, the number of distinct proofs of the theorem has augmented to exceed one hundred \parencite{Bogomolny_Pythagorean}. Drawing on a multitude of branches of mathematics, it can be established through both algebraic manipulation and visual rearrangement grounded in geometry. Particular compelling examples include the various proofs provided in the book \textit{Proofs Without Words} by \textcite{nelsen1993proofs}, demonstrating that the logical validity of mathematical conclusions is preserved despite the absence of conventional symbolic and linguistic modes of conveyance. While this insinuates a strong degree of insulation of epistemic power, it raises the question whether this independence is universal to all situations. Furthermore, epistemic power, as considered this far, concerns only the potential authority of knowledge, rather than a practical force.

When considered in a social context, the power of mathematical knowledge is no longer exclusively secured by its validity; instead, it depends on its accessibility and relevance within a community, emerging only when knowledge transitions from personal understanding to shared comprehension. Therefore, the social power of mathematical knowledge is conditional on how one imparts it. To enable the uptake and application of knowledge by others, it must be accessible, which in turn is dependent on the means of explication. Where conveyance is ineffective, knowledge may prove socially inert despite holding epistemic rigor. Fermat's Last Theorem clearly is emblematic of this, remaining inaccessible to all save a small number of scholars \parencite{Cambridge_FLT_History}. While the conclusion is accepted as justified belief, its social influence is near negligible, given the proof's complex and intricate nature \parencite{Boston_FLT_2003}. As \textcite{STYLIANOU2002303} argues in his paper regarding \textit{Representational Negotiation}, the means by which knowledge in mathematics is conveyed is highly dependent on the extent to which it is understood, especially beyond expert audiences. It therefore follows that epistemic certainty of knowledge proves insufficient for social power, which is shaped by the way it is communicated. Nevertheless, it must be noted that this influence is conditional upon the relevance to its audience. Where a body of knowledge bears no practical, contextual or intellectual significance for a crowd, even effective transmission fails to generate social power. This insight does not negate the effect of conveyance, it merely provides an extreme case in which the argument subtly wavers.

Mathematics reveals that epistemic power is mostly secured irrespective of how knowledge is conveyed, whereas social power is largely contingent on it. While neither claim holds universality, the identified exceptions serve to delimit rather than undermine their validity.

Rather than merely establish abstract truth, as is the case in Mathematics, the Human Sciences are intrinsically embedded within the societies they seek to investigate and influence, and thus, cannot be meaningfully considered void of their environment. Whereas in Mathematics the role of the onlooker is largely confined to verifying the antecedent conclusion, in the Human Sciences conclusions must be constructed by the individual through the interpretation and evaluation of empirical findings, rarely yielding single, absolute conclusions. As the uptake of information \textemdash\ and by extension the conclusion made and the knowledge assimilated \textemdash\ is shaped by how it is conveyed, both the epistemic and social power of knowledge in the Human Sciences are heavily reliant on the method of communication \parencite{Tversky1985}. This striking dependency became particularly visible during the COVID-19 pandemic: identical epidemiological data was communicated in manifold ways, underwriting differing claims and, in turn, generating a profound spectrum of public and political responses \parencite{Teschendorf09102024}. Consequently, both the epistemic and social power of the knowledge was molded by communicative representation, as public confidence in the credibility of information and the behavior of other members of the society was governed by the conveyance method. Moreover, to facilitate the communication of risk regarding the virus, the communicator must make axiological choices regarding the emphasis and framing of evidence \parencites{WuJohn2021RiskCommunication, Gigerenzer2007StatisticalLiteracy}. If these values fail to resonate with those of the recipient, the knowledge may manifest as latent, minimizing its social power. Hence, it is reasonable to conclude that the epistemic and social power of knowledge in the Human Sciences lies not simply in what is know, yet in how the knowledge is understood by the populace, highlighting its dependency on the communication method.

Naturally, this inference raises existential concern, in particular due to the omnipresent nature of the Human Sciences in our lives. To mitigate this shortcoming, Human-Science practitioners employ stringent methodological safeguards, seeking to anchor knowledge claims independent of their method of converse.

	
	
	
	

	
	
	
	
	
	
	\clearpage
	
	\printbibliography
	
	\clearpage
	
\end{document}